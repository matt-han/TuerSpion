%%
%% Beuth Hochschule für Technik 
%%
%% Einleitung 1
%%
%%

\chapter{Einleitung}
Der stetige technologische Fortschritt sorgt in allen Lebensbereichen für immer neue Erfindungen und Ideen. Eine der größten Entwicklungsbereiche für Privatanwender liegt im intelligenten Wohnen. Hierbei handelt es sich im Allgemeinen um technische Hilfsmittel, die einer Person oder einer Personengruppe das Leben erleichtert. Die Entwicklung dieser Geräte wird durch leistungsfähige Hardware sowie die mächtigen Entwicklungswerkzeuge ermöglicht. Die geringen Kosten und die umfangreiche Dokumentation im Internet machen es auch Privatanwendern möglich ihre eigenen Ideen zu verwirklichen.  
\par
Ein Teilbereich des intelligenten Wohnens ist die Überwachung, in den sich auch dieses Projekt einordnen lässt. Mit dem \textit{Spyhole} soll es möglich werden, Kontrolle und Sicherheit über die Eingangstür zu bekommen. Die Idee ist, von überall und jederzeit durch den Türspion seiner Wohnung schauen zu können. Weiterhin ist das ferngesteuerte Öffnen sowie das Abfragen der letzten Besucher eine erstrebenswerte Funktionalität. In Zeiten dauerhafter Vernetzung und der Smartphones steht es auch außer Frage, dass eine entsprechende Applikation für diese Systeme bereitgestellt werden muss. Es ist jedoch ebenso an Alternativsysteme zu denken, da es zur Projektvision gehört den Zugriff von überall zu ermöglichen. 
\par
Als Grundlage für dieses Projekt soll dabei hardwareseitig das Raspberry Pi verwendet werden, worauf in Kapitel zwei eingegangen wird. Um ein besseren Überblick des Projektes zu haben befindet sich im Anhang \ref{A.überblick} eine Übersicht der Komponenten und Struktur des Projektes. Das Konzept und die Umsetzung können dann in den folgenden Kapiteln gelesen werden.